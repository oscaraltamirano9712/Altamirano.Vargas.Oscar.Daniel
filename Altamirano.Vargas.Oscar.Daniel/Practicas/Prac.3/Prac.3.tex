\documentclass[12pt]{article}
\usepackage[T1]{fontenc}
\usepackage[utf8]{inputenc}
\usepackage[spanish]{babel}
\usepackage{graphicx}
\title{Practica 3}
\author{Cinemática de manipuladores seriales  \\ \\ Oscar Daniel Altamirano Vargas\\}

\begin{document} 
\maketitle
\begin{figure}[hbtp]
\centering
\includegraphics[width=10cm]{../../../FotosdeLatex/LOGO.jpg}
\end{figure}
\pagebreak
Para esta practica se realizaran unas cuantas operaciones y pasos a seguir vasandonos en las condiciones de D-H.\\ 

\includegraphics[width=10cm]{../../../FotosdeLatex/Captura18.PNG}  \\

$ \bullet $El paso numero uno consta en la asignación de sistemas de referencias de acuerdo a D-H.\\ \\ 

\includegraphics[width=8cm]{../../../FotosdeLatex/InkedCaptura18_LI.jpg} \\


$ \bullet $En este paso seria la obtención de la tabla de parámetros de D-H \\
\begin{figure}[hbtp]
\centering
\includegraphics[scale=1]{../../../FotosdeLatex/Captura19.PNG}\\ 
Como conclusión al seguir los pasos de D-H obtendremos matematicamente reflejado las posiciones de motores o articulaciones de un robot como se mostro en la tabla de parametros 
0
\end{figure}










\end{document}