\documentclass[12pt]{article}
\usepackage[T1]{fontenc}
\usepackage[utf8]{inputenc}
\usepackage[spanish]{babel}
\usepackage{graphicx}
\title{PRACTICA 2}
\author{Diceños CAD de robot \\ \\ Oscar Daniel Altamirano Vargas\\}

\begin{document} 
\maketitle
\begin{figure}[hbtp]
\centering
\includegraphics[width=10cm]{../../../FotosdeLatex/LOGO.jpg}
\end{figure}
\pagebreak

Para la elaboración de los diseños de piezas para el robot serial que se elaborará, será necesario contar con las medidas adecuadas para su simulación en los planos de CAD en 3D, así podremos representar en un montaje final el robot serial completo.\\
\\ \\
Previas piezas para su elavoración 
\begin{figure}[hbtp]
\caption{4-Husillos de 600mm y 2 de 300mm }
\centering
\includegraphics[width=11cm]{../../../FotosdeLatex/Captura14.PNG}
\end{figure} \\
\begin{figure}[hbtp]
\caption{Perfil de alminio de 20x20mm para la base del robot}
\centering
\includegraphics[width=11cm]{../../../FotosdeLatex/Captura15.PNG}
\end{figure}
\pagebreak
\begin{figure}[hbtp]
\caption{       3-Motores nema 17 de 1.2a}
\centering
\includegraphics[width=11cm]{../../../FotosdeLatex/Captura16.PNG}
\end{figure} \\
\begin{figure}[hbtp]
\caption{     4-guias lisas de 600mm y 2 de 300mm}
\centering
\includegraphics[width=11cm]{../../../FotosdeLatex/Captura17.PNG}
\end{figure}\\\\

\pagebreak
Al finalizar el montaje de cada pieza del robot se obtuvo el siguiente plano:\\
\begin{figure}[hbtp]
 \caption{Robot cartesiano}
 \centering
 \includegraphics[width=15cm]{../../../FotosdeLatex/Captura18.PNG}
 \end{figure}
  
\pagebreak
\end{document}