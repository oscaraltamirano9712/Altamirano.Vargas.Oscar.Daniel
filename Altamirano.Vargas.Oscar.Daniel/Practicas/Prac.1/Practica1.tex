\documentclass[12pt]{article}
\usepackage[T1]{fontenc}
\usepackage[utf8]{inputenc}
\usepackage[spanish]{babel}
\usepackage{graphicx} 
\usepackage{amsmath}
\title{PRACTICA 1}
\author{Reporte Istalacion de ROS\\ \\ \\ Oscar Daniel Altamirano Vargas\\}

\begin{document} 
\maketitle
\begin{figure}[hbtp]
\centering
\includegraphics[scale=1]{../../../FotosdeLatex/LOGO.jpg}
\end{figure}
\pagebreak
INSTALACIÓN Y CONFIGURACIóN DE SOURCE.LIST   \\
La configuración de su ordenador para aceptar el software de packages.ros.org. ROS índigo solo soporta Saucy (13.10) y Trusty (14.04) para los paquetes de Debian. \cite{INDIGO2014}
Source.list es un archivo se encuentra los paquetes y dependencias necesarios para instalar un software
Para agregar las librerías y dependencias necesarias al archivo source.list ejecuta la siguiente linea de comando:\\ \\
\includegraphics[scale=1]{../../../FotosdeLatex/Captura6.PNG} 
CONFIGURAR CLAVES Y ACTUALIZAR PAQUETES \\
Agregar la llave y firma para descargar los paquetes fiables del repositorio source.list con el comando:\\ \\
\includegraphics[scale=1]{../../../FotosdeLatex/Captura7.PNG} \\
INSTALAR PAQUETES DE ROS \\
La instalación de ROS índigo recomendada a instalar incluye bibliotecas robot-genérica, simuladores 2D / 3D, la navegación y la percepción 2D / 3D. (Se llevara un tiempo instalando ya que son 550 MB Aproximadamente) \\ \\
\includegraphics[scale=1]{../../../FotosdeLatex/Captura8.PNG} \\
INSTALAR ROSDEP\\
Antes de poder utilizar ROS, necesitará inicializar rosdep. Rosdep permite instalar fácilmente las dependencias del sistema de fuente que desea recopilar y necesario para ejecutar algunos componentes básicos en ROS.\\ \\
\includegraphics[scale=1]{../../../FotosdeLatex/Captura9.PNG} \\ \\
\includegraphics[scale=1]{../../../FotosdeLatex/Captura10.PNG} \\
CONFIGURACIÓN DEL ENTORNO \\
Es conveniente si las variables de entorno de ROS se añaden automáticamente a la sesión de golpe cada vez que se inicia una nueva shell:\\ \\
\includegraphics[scale=1]{../../../FotosdeLatex/Captura11.PNG} \\
OBTENER ROSINSTALL \\
\includegraphics[scale=1]{../../../FotosdeLatex/Captura12.PNG} 
\pagebreak
\nocite{*} 
\bibliographystyle{plain}
\bibliography{C:/ligas.JabRef/Prac.1/ROS}

\end{document}