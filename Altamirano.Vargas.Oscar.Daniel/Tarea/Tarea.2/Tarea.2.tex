\documentclass[12pt]{article}
\usepackage[T1]{fontenc}
\usepackage[utf8]{inputenc}
\usepackage[spanish]{babel}
\usepackage{graphicx} 
\usepackage{amsmath}
\title{TAREA 2}
\author{Parametros de rotaciones \\ de ángulos de Euler\\ \\ \\ Oscar Daniel Altamirano Vargas\\}

\begin{document} 
\maketitle
\begin{figure}[hbtp]
\caption{}
\centering
\includegraphics[scale=1]{../../../FotosdeLatex/LOGO.jpg}
\end{figure}
\pagebreak

Expresan una posicion mas general con un puntp fijo mediante 3 ángulos. La posición se alcanza mediante 3 rotaciones sucesivas. \\ \\
1-Precesión: Giro alrededor de un eje fijo: $\phi$ \\
2-Nutación: Giro alrededor del eje perpendicular al fijo y a otro solidario:$\theta$ \\
3-Rotación propia (spin): Giro alrededor del eje solidario al cuerpo: $\psi$ \\ \\ \\
\begin{figure}[hbtp]
\caption{}
\centering
\includegraphics[width=12cm]{../../../FotosdeLatex/Captura1.PNG}
\end{figure}

\pagebreak


PRIMER GIRO ÁNGULO DE PRECESIÓN\cite{cartagena992016}  ($\phi$) \\ \\
$
\begin{pmatrix}
i_1 \\
j_1 \\
k_1 \\
\end{pmatrix}
$
=
$
\begin{bmatrix}
cos \phi & sen\phi & 0\\
-sen\phi & cos\phi & 0\\
0 & 0 & 1 \\
\end{bmatrix} 
 $
$ \cdot $
$ 
\begin{pmatrix}
i \\
j \\
k 
\end{pmatrix}
$
\\ \\ \\
$
\begin{pmatrix}
i_1 \\
j_1 \\
k_1 \\
\end{pmatrix}
$
= Ez$ \phi $ $ \cdot $
$ 
\begin{pmatrix}
i \\
j \\
k 
\end{pmatrix}
$
\includegraphics[width=5cm]{../../../FotosdeLatex/Captura2.PNG}  
\\ \\ \\ 
SEGUNDO GIRO ÁNGULO DE NUTACIÓN ($ \theta $) \\ \\
$
\begin{pmatrix}
i_2 \\
j_2 \\
k_2 \\
\end{pmatrix}
$
=
$
\begin{bmatrix}
1 & 0 & 0 \\
0 & cos\theta  & sen\theta \\
0 & -sen\theta & cos\theta \\

\end{bmatrix} 
 $
$ \cdot $
$ 
\begin{pmatrix}
i_1 \\
j_1 \\
k_1 
\end{pmatrix}
$
\\ \\ \\
$
\begin{pmatrix}
i_2 \\
j_2 \\
k_2 \\
\end{pmatrix}
$
= Ex$ \theta $ $ \cdot $
$ 
\begin{pmatrix}
i_1 \\
j_1 \\
k_1 
\end{pmatrix}
$
\includegraphics[width=5cm]{../../../FotosdeLatex/Captura3.PNG} 
 
\pagebreak
TERCER GIRO ÁNGULO DE SPIN ($ \psi $) \\ \\
$
\begin{pmatrix}
e_1 \\
e_2 \\
e_3 \\
\end{pmatrix}
$
=
$
\begin{bmatrix}
cos\psi  & sen\psi & 0 \\
-sen\psi & cos\psi & 0 \\
0 & 0 & 1

\end{bmatrix} 
 $
$ \cdot $
$ 
\begin{pmatrix}
i_2 \\
j_2 \\
k_2 
\end{pmatrix}
$
\\ \\ \\
$
\begin{pmatrix}
e_1 \\
e_2 \\
e_3 \\
\end{pmatrix}
$
= Ez$ \psi $ $ \cdot $
$ 
\begin{pmatrix}
i_2 \\
j_2 \\
k_2 
\end{pmatrix}
$
\includegraphics[width=5cm]{../../../FotosdeLatex/Captura4.PNG}  \\ \\ \\
MATRIZ DE ROTACIÓN DE LOS 3 GIROS \\ \\
$
\begin{pmatrix}
i_1 \\
j_1 \\
k_1 \\
\end{pmatrix}
$
= Ez$ \phi $ $ \cdot $
$ 
\begin{pmatrix}
i \\
j \\
k 
\end{pmatrix}
$
$
\begin{pmatrix}
i_2 \\
j_2 \\
k_2 \\
\end{pmatrix}
$
= Ex$ \theta $ $ \cdot $
$ 
\begin{pmatrix}
i_1 \\
j_1 \\
k_1 
\end{pmatrix}
$
$
\begin{pmatrix}
e_1 \\
e_2 \\
e_3 \\
\end{pmatrix}
$
= Ez$ \psi $ $ \cdot $
$ 
\begin{pmatrix}
i_2 \\
j_2 \\
k_2 
\end{pmatrix}
$
\\
\\ 
\\
$
\begin{pmatrix}
i_2 \\
j_2 \\
k_2 \\
\end{pmatrix}
$
= Ex$ \theta $ $ \cdot $ Ez $ \phi $
$ 
\begin{pmatrix}
i \\
j \\
k 
\end{pmatrix}
$
\\
\\
\\
$ 
\begin{pmatrix}
e_1 \\
e_2\\
e_3 
\end{pmatrix}
$
=Ez$ \psi \cdot $Ex$ \theta\cdot $Ez$ \phi $
$ 
\begin{pmatrix}
i \\
j \\
k 
\end{pmatrix}
 $
 $ \Rightarrow $
 $ 
\begin{pmatrix}
e_1 \\
e_2\\
e_3
\end{pmatrix}
 $
 = E
 $ 
\begin{pmatrix}
i \\
j \\
k 
\end{pmatrix}
 $
 \\
 \includegraphics[scale=1]{../../../FotosdeLatex/Captura5.PNG} 

\pagebreak
\nocite{*} 
\bibliographystyle{plain}
\bibliography{C:/ligas.JabRef/Tarea.2/AngulosdeEuler}

\end{document}
