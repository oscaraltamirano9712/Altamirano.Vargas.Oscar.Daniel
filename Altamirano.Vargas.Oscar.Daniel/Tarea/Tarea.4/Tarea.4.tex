\documentclass[12pt]{article}
\usepackage[T1]{fontenc}
\usepackage[utf8]{inputenc}
\usepackage[spanish]{babel}
\usepackage{graphicx}
\title{TAREA 3}
\author{Condiciones de singularidad de manipuladores seriales \\ \\ Oscar Daniel Altamirano Vargas\\}

\begin{document} 
\maketitle
\begin{figure}[hbtp]
\centering
\includegraphics[width=10cm]{../../../FotosdeLatex/LOGO.jpg}
\end{figure}
\pagebreak
SINGULARIDAD\cite{monografia2015} \\ \\

$\bullet$ Un robot manipulador representa una configuración singular cuando el Jacobiano posee lineas que son linealmente dependientes. \\
$ \bullet $Se llama singularidad limite cuando el manipulador esta totalmente estendido o retraido. \\
$ \bullet $ Se llama singularidad interna cuando ocurre el alineaminento de dos o mas ejesde los sistemas de coordenadas, tomando las lineas del jacobiano linealmente dependientes, Este tipo de singularidad puede ocurrir en cualquier posición de actuador final.\\  \\ \\
-Es importante conocer las configuraciones de un robot por las siguientes cuestiones:\\ \\  \\
$ \bullet $ Causa pérdida de movilidad del robot.\\
$ \bullet $ Cuando el robot está en una configuración, pueden existir infinitas soluciones para la cinemática inversa. \\
$ \bullet $ Cuando el manipulador se aproxima a una configuración singular, una pequeña velocidad del actuador final provoca grandes velocidades en el accionamiento del robot.\\ \\ \\  
Las articulaciones de los manipuladores robóticos siguen diversas trayectorias para completar una tarea específica. La precisión en el movimiento de la estructura mecánica depende de las fuerzas aplicadas a las articulaciones del manipulador usando controladores. Existen varias técnicas de control lineal y no lineal usadas para el control articular de los manipuladores robóticos, considerando el manipulador como un sistema una-entrada/una-salida o single-input/single-output SISO, o como un sistema múltiples-entradas/múltiples-salidas o Multi-input/Multi-output MIMO. 
 

\pagebreak
\nocite{*} 
\bibliographystyle{plain}
\bibliography{C:/Ligas.JabRef/Tarea.4/tarea.4} \\

\end{document}