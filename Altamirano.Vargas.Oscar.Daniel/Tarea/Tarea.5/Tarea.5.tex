\documentclass[12pt]{article}
\usepackage[T1]{fontenc}
\usepackage[utf8]{inputenc}
\usepackage[spanish]{babel}
\usepackage{graphicx}
\usepackage{amsmath}
\title{TAREA 5}
\author{Operador Jacobiano \\ \\ Oscar Daniel Altamirano Vargas\\}

\begin{document} 
\maketitle
\begin{figure}[hbtp]
\centering
\includegraphics[width=10cm]{../../../FotosdeLatex/LOGO.jpg}
\end{figure}
\pagebreak
La matriz jacobiana es una matriz formada por las derivadas parciales de primer orden de una funcoón. Una de las aplicaciones más interesantes de esta matriz es la posibilidad de aproximar linealmente a la función en un punto. En este sentido, el jacobiano representa la derivada de una función multivariable.\cite{slideshare2012} \\ \\

Propiamente deneriamos hablar más que de matriz jacobiana, de diferencial jacobiana o aplicación lineal jacobiana ya que la forma de la matriz dependerá de la base o coordenadas elegidas. Es decir, dadas dos bases diferentes la aplicación lineal jacobiana tendrá componentes diferentes aún tratándose del mismo objeto matemático.\\ \\

Jac(f)=J(f)= $ \bigtriangledown $ $ \otimes $ f= $
\begin{pmatrix}
a/a_1 \\
a/a_1 \\
a/a_1 \\
\end{pmatrix}
$
$ \otimes $
$ (f_1,f_2,f_3)= $
$
\begin{pmatrix}
af_1/ax_1 | af_2/ax_1 | af_3/ax_1\\
af_1/ax_2 | af_2/ax_2 | af_3/ax_2\\
af_1/ax_3 | af_2/ax_3 | af_3/ax_3\\
\end{pmatrix}
$
\\ \\

el método jacobiano es simplemente un determinante que sirve para pasar o transformar de un sistema de coordenas a otro. Tambien se llama determinante funcional.
\pagebreak
\nocite{*} 
\bibliographystyle{plain}
\bibliography{C:/Ligas.JabRef/Tarea.5/tarea.5}
\end{document}