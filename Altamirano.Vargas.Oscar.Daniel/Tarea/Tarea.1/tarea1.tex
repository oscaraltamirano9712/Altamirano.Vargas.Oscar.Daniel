\documentclass[12pt]{article}
\usepackage[T1]{fontenc}
\usepackage[utf8]{inputenc}
\usepackage[spanish]{babel}
\usepackage{graphicx}
\title{TAREA 1}
\author{PAR DE ROTACION Y CUATERNIOS\\ \\ \\ Oscar Daniel Altamirano Vargas\\}

\begin{document} 
\maketitle
\begin{figure}[hbtp]

\centering
\includegraphics[scale=1]{../../LOGO.jpg}
\end{figure}

\pagebreak
El tema de rotacion habla sobre el desplazaiento o movimiento de el punto la cual se expresa con una matiz de (rotacion y traslacion) la cual representa la posición y orientación de un sistema girado y trasladado con respecto a "X,Y,Z". \\ \begin{figure}[hbtp]
\centering
\includegraphics[scale=1]{../../IMG-8701.JPG}
\caption{representación }
\end{figure}

Es necesario elavorar o tener en cuenta las matrices homogéneas básicas de rotación.\\
\begin{figure}[hbtp]
\caption{matrices}
\centering
\includegraphics[width=11cm]{../../IMG-8702.JPG}
\end{figure}

Los cuaternios matematicamante son utilizados para encontrar la relacion entre orientacion y rotacion entres dimensiones.\\
Los cuaternios son utilizados en aplicaciones gráficas por computadora, tobótica, navegación y mecánica es por eso que son necesarios dentro de la cinemática de un robot 


\end{document}
