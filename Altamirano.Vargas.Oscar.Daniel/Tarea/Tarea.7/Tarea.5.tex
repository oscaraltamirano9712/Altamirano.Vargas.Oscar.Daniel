\documentclass[12pt]{article}
\usepackage[T1]{fontenc}
\usepackage[utf8]{inputenc}
\usepackage[spanish]{babel}
\usepackage{graphicx}
\title{TAREA 7}
\author{Cinemática directa e inversa de manipuldores paralelos \\ \\ Oscar Daniel Altamirano Vargas\\}

\begin{document} 
\maketitle
\begin{figure}[hbtp]
\centering
\includegraphics[width=10cm]{../../../FotosdeLatex/LOGO.jpg}
\end{figure}
\pagebreak
CINEMÁTICA DIRECTA
\\ \\
La cinemática directa permite conocer cuál es la posición y orientación que adopta el extremo del robot cuando cada una de las variables que fijan la posición u orientación de sus articulaciones toma valores determinados.
También podremos mencionar que no es de interés saber las causas que se presentan en el movimiento de los motores. Si no, únicamente se toma en cuenta la descripción del mismo. \cite{we-robotica2013}
 \\ \\
Como ejemplo se muestra en la figura, un brazo de dos grados de libertad.\\ \\
\includegraphics[width=9cm]{../../../FotosdeLatex/Captura31.png} \\ \\ 
\includegraphics[width=9cm]{../../../FotosdeLatex/Captura35.jpg}  \\
Ejemplo de la cinemática directa de un BR de dos grados de libertad.\\
El análisis general que se realiza en la cinemática Directa es la siguiente:\\ \\

\includegraphics[width=10cm]{../../../FotosdeLatex/Captura32.png} \\ \\ \\
La cinemática inversa consiste en encontrar los valores que deben adoptar las coordenadas articulares del robot para que su extremo se posicione y oriente según una determinada localización espacial.  \\ \\

A diferencia de la cinemática Directa, el análisis general que se realiza es la siguiente: \\ \\

\includegraphics[width=10cm]{../../../FotosdeLatex/Captura33.png} \\ \\
\includegraphics[width=12cm]{../../../FotosdeLatex/Captura34.PNG} \\
\pagebreak
\nocite{*} 
\bibliographystyle{plain}
\bibliography{C:/Ligas.JabRef/Tarea.7/Tarea.7}





\end{document}